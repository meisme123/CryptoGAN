\documentclass{proc}

\usepackage{a4}
\usepackage{lipsum}
\usepackage{graphicx}
\usepackage{algorithmic}
\usepackage{cite}
\usepackage{xcolor}
\usepackage{amsmath,amssymb,amsfonts}
\usepackage{textcomp}

\usepackage{geometry}[margin=1.25in]

\title{A Review of Adversarial Neural Cryptography}
\author{
  {\bf Alisamar Husain}\\
  Dept. of Electrical Engineering, \\
  Jamia Millia Islamia
}

% \date{}

% \author{Alisamar Husain}
% \email{syed176412@st.jmi.ac.in}
% \affiliation{
%   \institution{Jamia Millia Islamia University}
%   \streetaddress{Maulana Mohammad Ali Jauhar Marg}
%   \city{New Delhi}
%   \state{Delhi}
%   \postcode{110025}
%   \country{India}
% }

\begin{document}
  \maketitle

  \begin{abstract}
    Artificial neural networks are well known for their ability to selectively 
    explore the solution space of a given problem.
    One of the recent applications of this feature is in the field of neural 
    cryptography, which provides an opportunity to use ANNs to encrypt data 
    such that it cannot be decrypted by an attacker.

    In this paper we examine the efficacy, feasibility and 
    general practicality of the use of {\it adversarial neural cryptography}, 
    as coined by Abadi et al. in \cite{seminalanc}, and neural cryptography
    in general. We test systems recommended in the literature and examine their 
    use in data transmission systems for the purpose of encrypting data from
    the perspective of securing a communication channel.
  \end{abstract}
 
  \section{Introduction}
  The field of cryptography is broadly concerned with algorithms and protocols 
  that ensure the secrecy and integrity of information. Cryptographic mechanisms are 
  typically described as programs or Turing machines. By this definition, an
  appropriate neural network can possibly be considered a cryptographic function.

    \subsection{Terminology}
    Certain terms are frequently used while talking about cryptographic mechanisms
    and it is beneficial to have an understanding of what these refer to. Some
    of these will be used in this paper to commonly identify certain parts of the 
    system and some are abbreviations made for convenience.

    A {\bf party} is a machine, or actor in general, which is using a communication
    channel to communicate with another machine. There are two major types of parties
    which we are concerned with, {\bf participants and attackers.}
    
    A {\bf participant} is a party which actively takes part in the communication and
    sends messages on the channel. The goal of encryption is to ensure that the 
    communication between any two parties can only be intercepted and understood 
    by them.

    An {\bf attacker} is a party which attempts to intercept and understand the
    communication between two participants.
  
  \paragraph{Attackers}
  Attackers are also described in those terms, with bounds on their complexity 
  (e.g., limite to polynomial time) and on their chances of success 
  (e.g., limited to a negligible probability). A mechanism is deemed secure if 
  it achieves its goal against all attackers. For instance, an encryption
  algorithm is said to be secure if no attacker can extract information about 
  plaintexts from ciphertexts.
  Modern cryptography provides rigorous versions of such definitions, like those 
  given by Goldwasser \& Micali. \cite{Goldwasser}

    \subsection{Symmetric Encryption}
    

  
  \section{Related Work}
  
  \section{Methodology}
  We use a simple setup in order to build and test the networks. The models are 
  first implemented using the Python programming language and after obtaining a 
  suitably trained and validated model, we can move on to testing. 

    \subsection{Tools Used}
    The models are implemented using PyTorch, a popular framework for building
    neural networks in Python. The models are built per the specification given in
    the literature and trained with the help of several common Python libraries.
    We use Tensorboard to monitor the training process and log the training metadata.

  
  \section{Results}
    \subsection{ANC}
    \subsection{Cryptonet}
  
  \section{Conclusions}

  \bibliographystyle{IEEEtran}
  % \bibliographystyle{acm}
  \bibliography{../../resources/citations}
\end{document}
